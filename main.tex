%!TEX program = xelatex -shell-escape
\documentclass[a4paper,12pt]{report}
\usepackage[margin=2.5cm]{geometry}
\usepackage{graphicx} % Required for inserting images
\usepackage{minted}
\usepackage{amsmath,amsfonts}
\newtheorem{definition}[chapter]{Definition}

\title{How do you solve a problem like the adjoint}
\author{David Ham}
\date{May 2023}

\begin{document}

\maketitle

\chapter{Functional setting}

\section{Parametrised multilinear forms}

Let $W_0\ldots W_{l-1}$, $V_0\ldots V_{k-1}$ be Hilbert spaces. The key examples of
such spaces to have in mind are $\mathbb{R}^d$ for positive integer $d$, and
(usually finite-dimensional) subspaces of the Sobolev spaces $L^2$, $H^n$,
$H(\operatorname{div})$ and $H(\operatorname{curl})$, defined over some
suitable domain $\Omega$.

\begin{definition}{Parametrised multilinear from}
    A $k$-linear form parametrised by $l$ coefficients is a function:
    \[
      f: W_{l-1} \times \ldots \times W_{0}; V_{k-1}\ldots V_{0} \rightarrow \mathbb{R}
    \]
    such that $f$ is constrained to be linear in the parameters corresponding
    to the $k$ spaces labelled $V$, but may be non-linear in the $l$ spaces
    labelled $W$. We call the parameters corresponding to the spaces labelled
    $V$ \emph{arguments}; and the parameters corresponding to the spaces
    labelled $W$ \emph{parameters}. 
\end{definition}
We further assume that at the point the problem is posed, the coefficients have
known values, while the value of the arguments is unknown.

As an example, consider the following finite element problem. 
Given $\nu\in W_0$, $f\in \hat{W}_0$, find $u \in V$ such that:
\begin{equation}
    \int_\Omega \nu\nabla u \cdot \nabla v\ \mathrm{d}x = 
    \int_\Omega f v\ \mathrm{d}x\qquad \forall v \in V
\end{equation}

The LHS of this problem is a bilinear form parametrised by 1 coefficient, given
by:
\begin{equation}
    a(\nu; u, v) = \int_\Omega \nu\nabla u \cdot \nabla v\ \mathrm{d}x
\end{equation}
while the RHS is a linear form parametrised by 1 coefficient, given by:
\begin{equation}
    L(f, v) = \int_\Omega fv\ \mathrm{d}x
\end{equation}

We conventionally write the lowest numbered argument last, so the test function
is argument 0 in each of these forms, while the trial function in the bilinear
form is argument 1, and the arguments are separated from the coefficients by a
semicolon.

\section{Gateaux Derivatives}

\begin{definition}{Gateaux derivative}
    Let $F: W_0\times \ldots \times W_{l-1}; V_0\ldots
    V_{k-1}:\rightarrow\mathbb{R}$ be a $k$-form parametrised by $l$
    coefficients. The \emph{Gateaux derivative} of $F$ with respect to the
    $m$-th coefficient $w_m$ is the $k+1$-form, $F: W_0\times \ldots \times
    W_{l-1}; W_m \times V_0\ldots V_{k-1}:\rightarrow\mathbb{R}$ given by:
    \[
        \frac{\partial f}{\partial w_m}(w_{l-1},\ldots,w_{0}; \hat{w}, v_k, \ldots v_0)
        = \lim_{\epsilon\rightarrow 0} \frac{
            f(w_{l-1},\ldots w_m + \epsilon \hat{w} \ldots w_0; \ldots)
            - f(w_{l-1},\ldots w_m \ldots w_0; \ldots)
        }{\epsilon}
    \]
\end{definition}

In simple cases this reduces to the definition of grad that you already know
from undergraduate vector calculus. For example, if $f: \mathbb{R}^2;\rightarrow R$
\begin{equation}
    nn
\end{equation}

\chapter{Reverse mode algorithmic differentiation}


\section{Algorithms and computer programs}

Algorithmic differentiation computes the derivative of a computer program with
respect to one or more of its inputs. It does so by capturing the underlying
algorithm and working backwards through the algorithm to compute the
derivative. We therefore need to start by understanding the relationship
between a program and its underlying variables.

Consider the following very simple Firedrake program:
\begin{listing}
    \inputminted[linenos]{python3}{examples/assembly.py}
    \caption{Firedrake code for a simple calculation and its gradient}
    \label{lst:assembly_code}
\end{listing}

This program computes the value $J$ via several intermediate calculations. It
then computes the derivative of $J$ with respect to the Firedrake function $f$.

At this stage we're not solving any PDEs, so we can duck that complication. Our
objective is to understand what this calculation means, what we mean by this
derivative, and how it's computed. We start by trying to understand the
computation.

\begin{figure}
    \includegraphics*[width=\textwidth]{examples/assembly.pdf}
    \caption{A Directed Acyclic Graph corresponding to the computation shown in
    listing \ref{lst:assembly_code}}. The variables \mintinline{python}+f+ and
    \mintinline{python}+g+ are shown as $w_4$ and $w_2$ respectively.
    \label{fig:assembly_dag}
\end{figure}

One way to think about the computation is as a Directed Acyclic Graph, as shown
in \ref{fig:assembly_dag}. This graph shows each computational operation as a
square box, and the values that are computed as ovals. There is a lot we can
already observe from this graph:

\begin{enumerate}
    \item Arrows connect operations to the values that are their inputs and
    outputs.
    \item Consequently, each arrow joins exactly one box and one oval.
    \item The same value of $f$ ($w_2$) is used twice, and hence the oval
    labelled $w_2$ is the source of two arrows.
    \item Conversely, changing the value of $g$ ($w_4$) produced a new oval,
    also labelled $w_4$.
\end{enumerate}

This last point uncovers a fundamental distinction between a computer program
in an imperative language such as Python, and a mathematical algorithm: some
computer program variables are mutable, their value can change during program
execution. Conversely, the variables of a mathematical algorithm are immutable:
their value does not change. Operations which would change a variable's value
instead create a new value. This is what happens here. The code:

\begin{minted}{python3}
    g.assign(2.0)
\end{minted}

changes the value of the Firedrake function $g$, but creates a new variable
(oval) in the DAG.

\section{Blocks and block variables}


We will adopt the terminology employed by pyadjoint, the library which provides
Firedrake's adjoint, to define the variables and operations in an algorithm.

\end{document}
